\documentclass[
11pt, % The default document font size, options: 10pt, 11pt, 12pt
%codirector, % Uncomment to add a codirector to the title page
]{charter} 

\usepackage{verbatim}


% El títulos de la memoria, se usa en la carátula y se puede usar el cualquier lugar del documento con el comando \ttitle
\titulo{Plataforma multipropósito de adquisición y visualización} 

% Nombre del posgrado, se usa en la carátula y se puede usar el cualquier lugar del documento con el comando \degreename
%\posgrado{Carrera de Especialización en Sistemas Embebidos} 
\posgrado{Carrera de Especialización en Internet de las Cosas} 
%\posgrado{Carrera de Especialización en Intelegencia Artificial}
%\posgrado{Maestría en Sistemas Embebidos} 
%\posgrado{Maestría en Internet de las cosas}

% Tu nombre, se puede usar el cualquier lugar del documento con el comando \authorname
\autor{Funes Pablo Nicolás} 

% El nombre del director y co-director, se puede usar el cualquier lugar del documento con el comando \supname y \cosupname y \pertesupname y \pertecosupname
\director{Gustavo Zocco}
\pertenenciaDirector{FIUBA} 
% FIXME:NO IMPLEMENTADO EL CODIRECTOR ni su pertenencia
\codirector{John Doe} % para que aparezca en la portada se debe descomentar la opción codirector en el documentclass
\pertenenciaCoDirector{FIUBA}

% Nombre del cliente, quien va a aprobar los resultados del proyecto, se puede usar con el comando \clientename y \empclientename
\cliente{Funes Pablo Nicolás}
\empresaCliente{Desarrollo personal}

% Nombre y pertenencia de los jurados, se pueden usar el cualquier lugar del documento con el comando \jurunoname, \jurdosname y \jurtresname y \perteunoname, \pertedosname y \pertetresname.
\juradoUno{Nombre y Apellido (1)}
\pertenenciaJurUno{pertenencia (1)} 
\juradoDos{Nombre y Apellido (2)}
\pertenenciaJurDos{pertenencia (2)}
\juradoTres{Nombre y Apellido (3)}
\pertenenciaJurTres{pertenencia (3)}
 
\fechaINICIO{18 de octubre de 2021}		%Fecha de inicio de la cursada de GdP \fechaInicioName
\fechaFINALPlan{6 de diciembre de 2021} 	%Fecha de final de cursada de GdP
\fechaFINALTrabajo{6 de diciembre de 2022}	%Fecha de defensa pública del trabajo final


\begin{document}

\maketitle
\thispagestyle{empty}
\pagebreak


\thispagestyle{empty}
{\setlength{\parskip}{0pt}
\tableofcontents{}
}
\pagebreak


\section*{Registros de cambios}
\label{sec:registro}


\begin{table}[ht]
\label{tab:registro}
\centering
\begin{tabularx}{\linewidth}{@{}|c|X|c|@{}}
\hline
\rowcolor[HTML]{C0C0C0} 
Revisión & \multicolumn{1}{c|}{\cellcolor[HTML]{C0C0C0}Detalles de los cambios realizados} & Fecha      \\ \hline
0      & Creación del documento                                 &\fechaInicioName \\ \hline
1      & Se completa hasta el punto 5 inclusive                 & 4 de noviembre de 2021 \\ \hline
2      & Se completa hasta el punto 9 inclusive y se realizan las correcciones de la entrega 1.
& 11 de noviembre de 2021 \\ \hline
3      & Se completa hasta el punto 12 inclusive y se realizan las correcciones de la entrega 2.
& 18 de noviembre de 2021 \\ \hline
4      & Se completa hasta el punto 15 inclusive y se realizan las correcciones de la entrega 3.
& 25 de noviembre de 2021 \\ \hline
%		  Se puede agregar algo más \newline
%		  En distintas líneas \newline
%		  Así                                                    & dd/mm/aaaa \\ \hline
%4      & Se completa hasta el punto 11 inclusive                & dd/mm/aaaa \\ \hline
%5      & Se completa el plan	                                 & dd/mm/aaaa \\ \hline
\end{tabularx}
\end{table}

\pagebreak



\section*{Acta de constitución del proyecto}
\label{sec:acta}

\begin{flushright}
Buenos Aires, \fechaInicioName
\end{flushright}

\vspace{2cm}

Por medio de la presente se acuerda con el Ing. \authorname\hspace{1px} que su Trabajo Final de la \degreename\hspace{1px} se titulará ``\ttitle'', consistirá esencialmente en la implementación de una plataforma genérica de adquisición y visualización, y tendrá un presupuesto preliminar estimado de 638 hs de trabajo y \$12.662, con fecha de inicio \fechaInicioName\hspace{1px} y fecha de presentación pública \fechaFinalName.

Se adjunta a esta acta la planificación inicial.

\vfill

% Esta parte se construye sola con la información que hayan cargado en el preámbulo del documento y no debe modificarla
\begin{table}[ht]
\centering
\begin{tabular}{ccc}
\begin{tabular}[c]{@{}c@{}}Ariel Lutenberg \\ Director posgrado FIUBA\end{tabular} & \hspace{2cm} & \begin{tabular}[c]{@{}c@{}}\clientename \\ \empclientename \end{tabular} \vspace{2.5cm} \\ 
\multicolumn{3}{c}{\begin{tabular}[c]{@{}c@{}} \supname \\ Director del Trabajo Final\end{tabular}} \vspace{2.5cm} \\
%\begin{tabular}[c]{@{}c@{}}\jurunoname \\ Jurado del Trabajo Final\end{tabular}     &  & \begin{tabular}[c]{@{}c@{}}\jurdosname\\ Jurado del Trabajo Final\end{tabular}  \vspace{2.5cm}  \\
%\multicolumn{3}{c}{\begin{tabular}[c]{@{}c@{}} \jurtresname\\ Jurado del Trabajo Final\end{tabular}} \vspace{.5cm}                                                                     
\end{tabular}
\end{table}


\section{1. Descripción técnica-conceptual del proyecto a realizar}
\label{sec:descripcion}

Hoy en día los seres humanos nos encontramos constantemente rodeados de información, la gran mayoría de las personas disponen de smartphones, dispositivos electrónicos y redes sociales. La información que hace un par de décadas era difícil de obtener, hoy se puede obtener en pocos segundos y alcance de la mano.\\
A modo de ejemplo, se pueden mencionar los smartwatch o relojes inteligentes como se muestra en la figura \ref{fig:pulsera_inteligente}. Los smartwatch se encuentran constantemente monitoreando al usuario y brindando información del usuario por medio de un display o una aplicación en el smartphone.\\
La información se encuentra disponible en una infinidad de sectores y en grandes volúmenes, se puede encontrar en sectores como una industria química, una papelera o incluso un centro de entrenamiento deportivo.
Los volúmenes de información requieren la necesidad de algún medio de discriminación para la información que nos resulta de interés. Para realizar la discriminación de la información se requieren técnicas de análisis y procesamiento de datos.\\
En todos los casos, disponemos de elementos capaces de recolectar la información, en el caso de la industria química podemos considerar sensores de ph, temperatura, posición y energía. De forma similar al ejemplo de la industria química, se podrían mencionar sensores para los ejemplos de la papelera o el centro de entrenamiento.\\
Un sensor se puede definir como un dispositivo capaz de detectar y responder a algún tipo de entrada del entorno físico. Los tipos de entrada pueden ser luz, calor, movimiento, humedad, presión o cualquiera de un gran número de otros fenómenos ambientales. La salida es generalmente una señal que se convierte en una pantalla legible por humanos en la ubicación del sensor o se transmite electrónicamente a través de una red para su lectura o procesamiento adicional. \\
El avance de la tecnología permitió el surgimiento de una infinidad de sensores con distintas capacidades físicas, almacenamiento y comunicación. A partir de esta característica, se puede mencionar que existe una gran cantidad de información dispersa en distintos elementos independientes.\\
 Hoy en día existen distintas aplicaciones que integran la información recopilada por los distintos sensores en una única plataforma, no obstante los costos de licencia, compatibilidad, usos de aplicación y el soporte son fundamentales para su funcionamiento.
 Las grandes industrias no tienen problemas de recursos al momento de realizar una transformación digital. Por el contrario pequeñas industrias, comercios, cooperativas entre otros, presentan recursos acotados y no disponen las capacidades para integrar dichos sistemas o no existen sistemas para el rubro en cuestión.\\
 El presente proyecto propone generar una plataforma para realizar una transformación digital o incluirse dentro de sectores que ya dispongan una integración digital.
La plataforma podrá ser utilizada en la mayor cantidad de escenarios posibles realizando configuraciones mínimas. Las configuraciones mínimas serán las imágenes de la aplicación, los servicios que recolectan la información de los sensores y las funcionalidades habilitadas para la aplicación.\\
El desarrollo de la plataforma se basa en separar el desarrollo en dos capas, una capa funcional común para cualquier problema y una capa de desarrollo particular para el cliente en cuestión.\\   
En la figura \ref{fig:diagBloques} se presenta el diagrama en bloques del sistema, se puede observar por un lado los clientes mobile/web y por otro lado el backend. Dentro del bloque de backend se puede distinguir los bloques denominados core genérico y workers. El bloque core genérico corresponde a la capa funcional común para los problemas como se menciono previamente, mientras que el bloque workers por su lado se corresponde con la lógica particular de los distintos problemas.\\
A su vez, en la figura \ref{fig:diagBloques}, se puede observar la existencia de un backoffice para realizar las configuraciones correspondientes para un cliente determinado desde un navegador web.\\
Para finalizar se propone un caso de aplicación, considerando un cliente hipotético que podría interesarle el proyecto.
En un centro de entrenamiento con un equipo profesional de futbol, se propone integrar la información  de los atletas utilizando un smartwatch en cada atleta. La información sera visualizada en una tablet por el entrenador o preparador físico. Por otro lado desde la propia aplicación, el encargado de las instalaciones podrá realizar acciones como configurar la iluminación y el riego.
A partir de esta situación, se realizaran las configuraciones sobre la plataforma para adaptarla a la necesidad del centro de entrenamiento.

\begin{figure}[htpb]
\centering 
\includegraphics[width=.55\textwidth]{./Figuras/pulsera2.jpg}
\caption{Smartwatch o reloj inteligente}
\label{fig:pulsera_inteligente}
\end{figure}

\begin{figure}[htpb]
\centering 
\includegraphics[width=.8\textwidth]{./Figuras/iot_nico.png}
\caption{Diagrama en bloques del sistema}
\label{fig:diagBloques}
\end{figure}

\section{2. Identificación y análisis de los interesados}
\label{sec:interesados}


\begin{table}[ht]
%\caption{Identificación de los interesados}
%\label{tab:interesados}
\begin{tabularx}{\linewidth}{@{}|l|X|X|l|@{}}
\hline
\rowcolor[HTML]{C0C0C0} 
Rol           & Nombre y Apellido & Organización 	& Puesto 	\\ \hline
%Auspiciante   &                   &              	&        	\\ \hline
Cliente       & Pablo Valentín Funes      & Funes - Ceriale Consultores en Ingeniería	& Ingeniero eléctrico\\ \hline
%Impulsor      &                   &              	&        	\\ \hline
Responsable   & \authorname       & FIUBA        	& Alumno 	\\ \hline
%Colaboradores &                   &              	&        	\\ \hline
Orientador    & \supname	      & \pertesupname 	& Director Trabajo final \\ \hline
%Equipo        & miembro1 \newline 
%				miembro2          &              	&        	\\ \hline
%Opositores    &                   &              	&        	\\ \hline
%Usuario final &                   &              	&        	\\ \hline
\end{tabularx}
\end{table}


\section{3. Propósito del proyecto}
\label{sec:proposito}

El propósito de este proyecto es generar una plataforma IOT multipropósito. La plataforma servirá de base para futuros trabajos de ingeniería, proporcionándole al alumno los conocimientos básicos para desarrollarse en los proyectos que puedan surgir.\\
Estos trabajos consistirían en la configuración de la plataforma para el cliente, con la posibilidad de integrar los distintos equipos en la plataforma como un servicio de ingeniería o un servicio de inteligencia y análisis de datos. 

\section{4. Alcance del proyecto}
\label{sec:alcance}

El proyecto incluye el desarrollo de un sistema encargado de:
\begin{enumerate}
\item Proveer una interfaz de usuario mobile (ios/android).
\item Proveer una interfaz de usuario web.
\item Proveer una interfaz de configuración web back office para administrar las funcionalidades según el cliente.
\item Registrar de forma persistente la información obtenida.
\item Generar alertas según parámetros establecidos.
\item Registrar las mediciones realizadas por sensores y actuadores.
\item Desarrollo del firmware en nodos sensores/actuadores para banco de pruebas.
\item Desarrollo del hardware en nodos sensores/actuadores para banco de pruebas.
\item Pruebas sobre la plataforma utilizando sensores de temperatura/humedad DHT22 como variables de entrada y 2 leds cumpliendo el rol de variables de salida.
\end{enumerate}

El proyecto no incluye:
\begin{enumerate}
\item Contratación de servicios de terceros.
\item Desarrollo de hardware de adquisición de datos.
\item Desarrollo de firmware para el hardware de adquisición de datos.
\item Desarrollos para las plataformas de Apple.
\end{enumerate}

\section{5. Supuestos del proyecto}

Para el desarrollo del presente proyecto se realizan los siguientes supuestos:
\begin{itemize}
	\item Se dispondrán los recursos económicos suficientes para adquirir los componentes necesarios para el banco de pruebas.
	\item Se dispondrá de los recursos necesarios para realizar la investigación de los recursos a emplear durante el proyecto.
	 \item Se dispondrá el apoyo de los orientadores a lo largo del proyecto.
	 \item Se dispondrá un alcance de las funcionalidades acorde al tiempo estipulado del proyecto (1 año).
\end{itemize}

\section{6. Requerimientos}
\label{sec:requerimientos}

A continuación se muestran los requerimientos del proyecto clasificados según el componente que tengan en común y la prioridad.
A menor valor numérico, la prioridad es mayor.

\begin{enumerate}
	\item Requerimientos interfaz cliente (web-mobile)
		\begin{enumerate}
			\item El sistema debe funcionar en los clientes web chrome y mozilla.
			\item El sistema debe funcionar en ios/android. 
			\item El usuario debe poder listar todos los sensores disponibles.
			\item El usuario debe poder visualizar las mediciones de los sensores disponibles.
			\item El usuario debe poder accionar un actuador en caso de contar con los permisos.
		\end{enumerate}
	\item Requerimientos del backoffice
		\begin{enumerate}
			\item El usuario debe poder ingresar mediante un login.
			\item El sistema debe permitir la configuración de las funcionalidades disponibles.
			\item El sistema debe permitir la actualización de los parámetros de un usuario.
			\item El sistema debe permitir la actualización de los parámetros de un sensor.
			\item El sistema debe permitir la configuración de una alerta a partir de los parámetros de un sensor.
			\item Requerimiento 2 (prioridad menor)
		\end{enumerate}
	\item Requerimientos del backend
		\begin{enumerate}
			\item El sistema debe persistir la información de los usuarios.
			\item El sistema debe persistir la información de los sensores.
			\item El sistema debe poder informar con una alerta si es necesario.
			\item El sistema debe poder comunicarse con los protocolos http-mqtt.
		\end{enumerate}
	\item Requerimiento de testing.
		\begin{enumerate}
			\item El sistema debe disponer una suite de pruebas unitarias para asegurar el funcionamiento.
			\item El sistema debe disponer un banco de pruebas con sensores y actuadores.
			\item El sistema debe disponer una serie de pruebas de integración utilizando el banco de pruebas.
		\end{enumerate}
\end{enumerate}

\section{7. Historias de usuarios (\textit{Product backlog})}
\label{sec:backlog}

En esta sección se incluyen las historias de usuarios y sus correspondientes ponderaciones.

\subsection{Usuarios}
Como usuario quiero registrarme en la plataforma e ingresar con mi correo electrónico para garantizar la seguridad de mi información.
\begin{itemize}
	\item Dificultad: 3 - El registro del usuario involucra muchas horas para asegurar la autenticación del mismo.
	\item Complejidad: 8 - La integración con algún servicio externo puede volverse difícil y deben considerarse muchos casos borde.
	\item Riesgo: 5 - Se debe asegurar el mecanismo para corroborar la identidad al momento del registro del correo electrónico, generalmente se usan sms o email.
\end{itemize}
Story Points: 16

\subsection{Sensores disponibles}
Como usuario quiero poder visualizar el listado completo de los sensores disponibles para poder seleccionar el sensor de interés.
\begin{itemize}
	\item Dificultad: 3 - Involucra hacer el pedido de información al backend y seleccionar la forma de mostrar la información.
	\item Complejidad: 5 - Hacer un servicio que devuelva información contemplando todos los casos posibles.
	\item Riesgo: 2 - Se puede dar el caso de no obtener información con el servicio teniendo que manejar un caso excepcional.
\end{itemize}

Story Points: 10
\subsection{Mediciones de un sensor}
Como usuario quiero poder visualizar las mediciones de un sensor seleccionado para observar el correcto funcionamiento del sistema.
\begin{itemize}
	\item Dificultad: 3 - Involucra el pedido de información al backend y seleccionar la forma de mostrar la información.
	\item Complejidad: 5 - Hacer un servicio que devuelva información contemplando todos los casos posibles.
	\item Riesgo: 3 - Se puede dar el caso de no obtener información con el servicio teniendo que manejar un caso excepcional.
\end{itemize}

Story Points: 11
\subsection{Alarmas}
Como usuario quiero recibir notificaciones por la aplicación o correo electrónico en el caso de una alarma para conocer el estado actual del sistema.
\begin{itemize}
	\item Dificultad: 5 - Se requiere correr un proceso de fondo revisando la situación particular de cada sensor para accionar una alarma.
	\item Complejidad: 13 - Cada sensor es diferente por lo que cada alarma es diferente y requiere darle un comportamiento particular.
	\item Riesgo: 5 - Puedo que no se dispare una alarma y ocurra un problema.
\end{itemize}

Story Points: 23
\subsection{Información de los sensores}
Como usuario quiero recibir la información de los sensores por correo electrónico en formato csv para poder realizar un posterior análisis de datos.
\begin{itemize}
	\item Dificultad: 5 - Requiere la búsqueda de la información particular y el formato.
	\item Complejidad: 5 - Requiere enviar un archivo por un servicio externo.
	\item Riesgo: 3 - El archivo puede contener errores de formato y no enviarse.
\end{itemize}

Story Points: 13
\section{8. Entregables principales del proyecto}
\label{sec:entregables}


Los principales entregables del proyecto son:

\begin{itemize}
	\item Manual de uso 
	\item Video tutorial de uso.
	\item Software ejecutable de la aplicación.
	\item Informe de avance del proyecto.
	\item Memoria del proyecto.
\end{itemize}


\section{9. Desglose del trabajo en tareas}
\label{sec:wbs}


En esta sección se muestra el listado de tareas que forman parte del proyecto.
\begin{enumerate}
\item Planificación del proyecto (70 hs)
	\begin{enumerate}
	\item Realizar el plan del proyecto (40 hs)
	\item Realizar el estudio de viabilidad económica-financiera del proyecto (20 hs)
	\item Aprobación de la planificación(10 hs)
	\end{enumerate}
\item Investigación y selección de tecnologías. (56 hs)
	\begin{enumerate}
	\item Investigación y selección de las tecnologías en el desarrollo del software (32 hs)
	\item Investigación de protocolos http-mqtt (8 hs)
	\item Investigacion sobre herramientas de deploy(16 hs)
	\end{enumerate}
\item Configuración de la infraestructura del proyecto. (58 hs)
	\begin{enumerate}
	\item Configuración de las bases de datos (8 hs)
	\item Configuración de las instancias de backend (12 hs)
	\item Configuración de las instancias de frontend web (16 hs)
	\item Configuración de un servicio externo para las pruebas unitarias CircleCi o Travis (6 hs).
	\item Configuración de herramientas para el deploy de la aplicación (16 hs).
	\end{enumerate}
\item Desarrollo del backend. (72 hs)
	\begin{enumerate}
	\item Desarrollo del endpoint del listado de sensores (8 hrs)
	\item Desarrollo del endpoint de las mediciones de un sensor  (8 hs)
	\item Desarrollo del endpoint para el registro de un usuario (12 hs)
	\item Desarrollo del endpoint para el login de un usuario  (8 hs)
	\item Desarrollo del backoffice (16 hs)
	\item Desarrollo del software para comunicacion con los sensores (20 hs)
	\end{enumerate}
\item Desarrollo del frontend web. (70 hs)
	\begin{enumerate}
	\item Diseño de las vistas de la aplicación (32 hs)
	\item Desarrollo de la vista del login de un usuario (12 hs)
	\item Desarrollo de la vista de un menú principal con las funcionalidades (8 hs)
	\item Desarrollo de la vista del listado de sensores (12 hs)
	\item Desarrollo de la vista de las mediciones de un sensor (12 hs)
	\end{enumerate}
\item Desarrollo del frontend mobile. (70 hs)
	\begin{enumerate}
	\item Diseño de las vistas de la aplicación (32 hs)
	\item Desarrollo de la vista del login de un usuario (12 hs)
	\item Desarrollo de la vista de un menú principal con las funcionalidades (8 hs)
	\item Desarrollo de la vista del listado de sensores (12 hs)
	\item Desarrollo de la vista de las mediciones de un sensor (12 hs)
	\end{enumerate}
\item Testing. (60 hs)
	\begin{enumerate}
	\item Diseño de casos de prueba (8 hs)
	\item Generación de datos de prueba (4 hs)
	\item Diseño del hardware de pruebas (8 hs)
	\item Ejecución del esquema de pruebas (34 hs)
	\item Generar reporte de las pruebas (6 hs)
	\end{enumerate}
\item Validación. (40 hs)
	\begin{enumerate}
	\item Diseño de ensayos de validación (12 hrs)
	\item Ensayos de validación (28 hs)
	\end{enumerate}
\item Presentación del trabajo. (142 hs)
	\begin{enumerate}
	\item Redacción del informe de avance (10 hs)
	\item Redacción del tutorial de uso (8 hs)
	\item Redacción de la memoria escrita (100 hs)
	\item Elaboración de presentación final (24 hs).
	\end{enumerate}	
\end{enumerate}

El tiempo total estimado del proyecto es de 638 hs.


\section{10. Diagrama de Activity On Node}
\label{sec:AoN}

En esta sección se muestra el diagrama de Activity On Node del proyecto.

\begin{figure}[htpb]
\centering 
\includegraphics[width=1\textwidth]{./Figuras/referencias_aon.png}
\caption{Referencias del diagrama \textit{Activity on Node}.}
\label{fig:AoN_referencias}
\end{figure}


\begin{figure}[htpb]
\centering 
\includegraphics[angle=270,width=0.9\textwidth]{./Figuras/aon.png}
\caption{Diagrama \textit{Activity on Node} expresado en horas.}
\label{fig:AoN}
\end{figure}


\section{11. Diagrama de Gantt}
\label{sec:gantt}

En esta sección se muestra el diagrama de Gantt del proyecto.

\begin{figure}[htpb]
\centering 
\includegraphics[angle=0,height=0.7\textheight]{./Figuras/planner_tareas.png}
\caption{Listado de tareas del diagrama de Gantt }
\label{fig:diagGantt_tareas}
\end{figure}

\begin{landscape}
\begin{figure}[htpb]
\centering 
\includegraphics[angle=0,height=0.86\textheight]{./Figuras/gantt.png}
\caption{Diagrama de Gantt}
\label{fig:diagGantt}
\end{figure}
\end{landscape}


\section{12. Presupuesto detallado del proyecto}
\label{sec:presupuesto}
En esta sección se muestra el presupuesto del proyecto.

\begin{figure}[htpb]
\centering 
\includegraphics[width=1.0\textwidth]{./Figuras/presupuesto.png}
\caption{Presupuesto del proyecto en dólares.}
\label{fig:presupuesto}
\end{figure}

\section{13. Gestión de riesgos}
\label{sec:riesgos}
En esta sección se describen los riesgos asociados al proyecto, considerando las medidas necesarias para reducir la probabilidad de ocurrencia y el impacto del riesgo.

Riesgo 1: Imposibilidad de adquirir los componentes del proyecto.
\begin{itemize}
	\item Severidad (9): Considero que sin los componentes no se puede realizar la mayor parte del proyecto.
	\item Probabilidad de ocurrencia (3): Considero que los componentes seleccionados se pueden adquirir sin problema por la generalidad de los mismos.
\end{itemize}


Riesgo 2: Imposibilidad de cumplir con los plazos de entrega.
\begin{itemize}
	\item Severidad (6):
	Considero que la planificación temporal realizada proporciona un margen de tiempo en caso de retrasos.
	\item Probabilidad de ocurrencia (5): 
	Considero que mi disponibilidad para trabajar en el proyecto puede variar debido a situaciones como: cambio de trabajo, enfermedades y la pandemia. 
\end{itemize}


Riesgo 3: Acceso a la información de la plataforma por personal no autorizado.
\begin{itemize}
	\item Severidad (7): Considero que el acceso a la información en ciertos rubros puede generar perdidas de gran relevancia.
	\item Probabilidad de ocurrencia (6): Considero que en la actualidad existen técnicas para extraer información de las plataformas digitales. Las técnicas evolucionan continuamente.\\
\end{itemize}


Riesgo 4: Fallas en los sensores.
\begin{itemize}
	\item Severidad (8): Considerando el monitoreo de un proceso critico, la visibilidad de información incorrecta o ausencia de información puede provocar la toma de decisiones erróneas con consecuencias. 
	\item Probabilidad de ocurrencia (5): Considero por experiencia que siempre pueden ocurrir fallas con los sensores por hardware o comunicación. 
\end{itemize}


Riesgo 5: Solución no escalable en el tiempo.
\begin{itemize}
	\item Severidad (8): Considero que al tratarse de un desarrollo genérico debe poder ser escalable en el tiempo.
	\item Probabilidad de ocurrencia (6): Considero que la elección de tecnologías y el modelo del sistema pueden posibilitar este riesgo.
\end{itemize}


b) Tabla de gestión de riesgos:      (El RPN se calcula como RPN=SxO)

\begin{table}[htpb]
\centering
\begin{tabularx}{\linewidth}{@{}|X|c|c|c|c|c|c|@{}}
\hline
\rowcolor[HTML]{C0C0C0} 
Riesgo & S & O & RPN & S* & O* & RPN* \\ \hline
Imposibilidad de adquirir los componentes del proyecto       &9   &3   &27     &*    &*    &*      \\ \hline
Imposibilidad de cumplir con los plazos de entrega       &6   &5   &30     &*    &*    &*      \\ \hline
Acceso a la información de la plataforma por personal no autorizado       &7   &6   &42     &6    &4    &24      \\ \hline
Fallas en los sensores       &8   &5   &40     &4    &5    &20      \\ \hline
Solución no escalable en el tiempo       &8  &6   &48     &8    &4    &32      \\ \hline
\end{tabularx}%
\end{table}

Criterio adoptado: 
Se tomarán medidas de mitigación en los riesgos cuyos números de RPN sean mayores a 36

Nota: los valores marcados con (*) en la tabla corresponden luego de haber aplicado la mitigación.

c) Plan de mitigación de los riesgos que originalmente excedían el RPN máximo establecido:
 
Riesgo 3: Acceso a la información de la plataforma por personal no autorizado.\\
Medidas de mitigación: Realizar una investigación sobre identificación de usuarios y realizar una implementación acorde. Realizar una investigación sobre que información podría ser accedida sin consentimiento.\\
  - Severidad (6): La severidad disminuye por acción de verificar el tipo de información que podría ser accedida.\\
  - Probabilidad de ocurrencia (4): La probabilidad de ocurrencia disminuye por la implementación de algún método de identificación de un usuario.

Riesgo 4: Fallas en los sensores.\\
Medidas de mitigación: Notificar por medio de la interfaz grafica el problema con el sensor.\\
  - Severidad (4): La severidad disminuye por el uso de notificaciones .\\
  - Probabilidad de ocurrencia (5): La probabilidad de ocurrencia se mantiene, los sensores en principio estarían fuera de alcance.
  
Riesgo 5: Solución no escalable en el tiempo.\\
Medidas de mitigación: Realizar un buen modelado de las entidades, selección de patrones de diseño y tecnologías.\\
  - Severidad (8): La severidad se mantiene.\\
  - Probabilidad de ocurrencia (4): La probabilidad de ocurrencia disminuye por la investigación sobre tecnologías y el diseño de una buena arquitectura.



\section{14. Gestión de la calidad}
\label{sec:calidad}

Para cada uno de los requerimientos del proyecto indique:
\begin{itemize} 
\item Req \#1.1: El sistema debe funcionar en los clientes web chrome y mozilla.
\begin{itemize}
	\item Verificación del desarrollo de frontend con los clientes web chrome y mozilla en la computadora de desarrollo. 
	\item Validación con el cliente con los dos clientes web utilizando la computadora del cliente o la del desarrollador.
\end{itemize}

\item Req \#1.2: El sistema debe funcionar en ios/android.
\begin{itemize}
	\item Verificación de las versiones mobile en los simuladores correspondientes y en los dispositivos móviles para el desarrollo. 
	\item Validación con el cliente utilizando el teléfono del cliente o el del desarrollador.
\end{itemize}

\item Req \#1.3: El usuario debe poder listar todos los sensores disponibles.
\begin{itemize}
	\item Verificación de la funcionalidad con pruebas unitarias e integración. 
	\item Validación con el hardware diseñado para testing.
\end{itemize}

\item Req \#1.4: El usuario debe poder visualizar las mediciones de los sensores disponibles.
\begin{itemize}
	\item  Verificación de la funcionalidad con pruebas unitarias e integración.
	\item Validación con el hardware diseñado para testing.
\end{itemize}

\item Req \#1.5: El usuario debe poder accionar un actuador en caso de contar con los permisos.
\begin{itemize}
	\item  Verificación de la funcionalidad con pruebas unitarias e integración.
	\item Validación con el hardware diseñado para testing.
\end{itemize}

\item Req \#2.1: El usuario debe poder ingresar mediante un login.
\begin{itemize}
	\item  Verificación de la funcionalidad con pruebas unitarias e integración.
	\item Validación utilizando los clientes web y mobile con el cliente.
\end{itemize}

\item Req \#2.2: El sistema debe permitir la configuración de las funcionalidades disponibles.
\begin{itemize}
	\item  Verificación de la funcionalidad con pruebas unitarias e integración.
	\item Validación utilizando los clientes web y mobile con una prueba diseñada puntualmente para este escenario particular.
\end{itemize}

\item Req \#2.3: El sistema debe permitir la actualización de los parámetros de un usuario.
\begin{itemize}
	\item  Verificación de la funcionalidad con pruebas unitarias e integración.
	\item Validación utilizando los clientes web y mobile con el cliente.
\end{itemize}

\item Req \#2.4: El sistema debe permitir la actualización de los parámetros de un sensor.
\begin{itemize}
	\item  Verificación de la funcionalidad con pruebas unitarias e integración.
	\item Validación utilizando los clientes web y mobile.
\end{itemize}

\item Req \#2.5: El sistema debe permitir la configuración de una alerta a partir de los parámetros de un sensor.
\begin{itemize}
	\item  Verificación de la funcionalidad con pruebas unitarias e integración.
	\item Validación utilizando las notificaciones de los clientes web y mobile.
\end{itemize}

\item Req \#3.1: El sistema debe persistir la información de los usuarios.
\begin{itemize}
	\item  Verificación de la funcionalidad con pruebas unitarias.
	\item Validación utilizando Postman, repitiendo la prueba en diferentes momentos.
\end{itemize}

\item Req \#3.2: El sistema debe persistir la información de los sensores.
\begin{itemize}
	\item  Verificación de la funcionalidad con pruebas unitarias.
	\item  Validación utilizando Postman, repitiendo la prueba en diferentes momentos.
\end{itemize}

\item Req \#3.3: El sistema debe poder informar con una alerta si es necesario.
\begin{itemize}
	\item  Verificación de la funcionalidad con pruebas unitarias.
	\item Validación utilizando Postman, repitiendo la prueba en diferentes momentos.
\end{itemize}

\item Req \#3.4: El sistema debe poder comunicarse con los protocolos http-mqtt.
\begin{itemize}
	\item  Verificación de la funcionalidad con pruebas unitarias.
	\item Validación utilizando Postman utilizando el hardware de testing.
\end{itemize}

\item Req \#4.1: El sistema debe disponer una suite de pruebas unitarias para asegurar el funcionamiento.
\begin{itemize}
	\item  Verificación por parte del desarrollador en el uso de TDD.
	\item Validación utilizando una métrica de testing coverage.
\end{itemize}

\item Req \#4.2: El sistema debe disponer un banco de pruebas con sensores y actuadores.
\begin{itemize}
	\item  Verificación sobre el hardware de testing seleccionado.
	\item Validación sobre el hardware de testing.
\end{itemize}

\item Req \#4.3: El sistema debe disponer una serie de pruebas de integración utilizando el banco de pruebas.
\begin{itemize}
	\item  Verificación de la existencia del plan de pruebas.
	\item Validación sobre los alcances y objetivos de las pruebas del plan.
\end{itemize}

\end{itemize}



\section{15. Procesos de cierre}    
\label{sec:cierre}

La ultima etapa del proyecto consiste en una presentación con la presencia de los jurados y el director del proyecto donde se hará énfasis en las siguientes actividades:

\begin{itemize}
	\item Pautas de trabajo que se seguirán para analizar si se respetó el Plan de Proyecto original:\\
	Encargado: Funes Pablo Nicolás\\
	\begin{itemize}
	 \item Se realizara una evaluación de los requerimientos y objetivos alcanzados considerando el plan de trabajo original.
	 \item Se realizara un análisis de causas sobre las diferencias mencionadas en el punto previo.
	 \end{itemize}
	\item Identificación de las técnicas y procedimientos útiles e inútiles que se emplearon, y los problemas que surgieron y cómo se solucionaron:
\\
	Encargado: Funes Pablo Nicolás\\
		\begin{itemize}
	\item Se realizara una evaluación sobre las distintas herramientas empleadas en el proyecto, haciendo énfasis en la utilidad de las mismas.
		\end{itemize}

	\item Indicar quién organizará el acto de agradecimiento a todos los interesados, y en especial al equipo de trabajo y colaboradores:
	\\
	Encargado: Funes Pablo Nicolás\\
		\begin{itemize}
	\item Al finalizar la defensa del proyecto, se procederá a agradecer a todos los involucrados en el proyecto.
	  	\end{itemize}
\end{itemize}



\end{document}
